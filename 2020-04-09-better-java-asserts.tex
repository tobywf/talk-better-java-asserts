%!TEX TS-program = XeLaTeX
\documentclass[xetex,12pt,aspectratio=169]{beamer}

\usepackage[english]{babel}
\usepackage{fontspec}
\setmainfont[Mapping=tex-text]{Source Serif Pro}
\setsansfont[Mapping=tex-text]{Source Sans Pro}
\setmonofont[Mapping=,Scale=0.85]{Source Code Pro}

\beamertemplatenavigationsymbolsempty

\definecolor{offwhite}{HTML}{fcfcfc}
\definecolor{lightgrey}{HTML}{848484}
\definecolor{mediumgrey}{HTML}{444453}
\definecolor{midgrey}{HTML}{222226}
\definecolor{darkgrey}{HTML}{18191c}

\definecolor{pink}{HTML}{F92672}

\setbeamercolor*{palette primary}{fg=midgrey,bg=offwhite}
\setbeamercolor*{normal text}{use=palette primary,fg=palette primary.fg,bg=palette primary.bg}
\setbeamercolor*{structure}{use=palette primary,fg=palette primary.fg,bg=palette primary.bg}

\usepackage{minted}

\usepackage{tikz}
\usetikzlibrary{calc}

\newcommand{\sectionframe}[1]{\begin{frame}[plain]\usebeamercolor{palette primary}
\begin{tikzpicture}[remember picture,overlay]
\fill [fg]  (current page.north east) rectangle (current page.south west);
\path (current page.north east) -- +(-1,-1) coordinate (northwest);
\path (current page.south west) -- +(+1,+1) coordinate (southwest);
\draw[bg] (northwest) rectangle (southwest) node [midway,align=center] {\usebeamerfont{title}#1};
\end{tikzpicture}
\end{frame}}

\AtBeginSection[]{\sectionframe{\insertsectionhead}}

\defbeamertemplate*{title page}{customized}[1][]
{\usebeamercolor{palette primary}
\begin{tikzpicture}[fg,remember picture,overlay]
\path (current page.north) -- +(-1,0) coordinate (north);
\path (current page.south) -- +(-1,0) coordinate (south);
\path (current page.south) -- +(-4,0) coordinate (southskew);
\fill (current page.north west) -- (north) -- (southskew) -- (current page.south west) -- cycle;
\node () at ($(north)!0.4!(south)$) [right] {\usebeamerfont{title}\inserttitle};
\node () at ($(north)!0.46!(south)$) [right] {\usebeamerfont{subtitle}\insertsubtitle};
\node () at ($(north)!0.6!(south)$) [right] {\usebeamerfont{author}\insertauthor};
\node () at ($(north)!0.7!(south)$) [right] {\usebeamerfont{date}\insertdate};
\end{tikzpicture}}

\title{Better Java Asserts}
\subtitle{}
\author{Toby Fleming}

\usepackage{verbatim}

\begin{document}

\maketitle

\section{What does a bad assert look like?}

\begin{frame}[fragile]{Facile example}
\begin{minted}{java}
import static org.junit.Assert.assertTrue;
assertTrue(a.equalsIgnoreCase(b));
\end{minted}
\vfill
\begin{verbatim}
java.lang.AssertionError
    at org.junit.Assert.fail(Assert.java:86)
    ...
\end{verbatim}
\vfill
\begin{center}
\includegraphics[height=48pt]{glyph-u1F62D.png}\includegraphics[height=48pt]{glyph-u1F62D.png}\includegraphics[height=48pt]{glyph-u1F62D.png}
\end{center}
\end{frame}

\begin{frame}[fragile,t]{Different example}
realistic example:
\vfill
\begin{minted}{java}
final Result a = new API().getResult();
final Result b = new Result.Builder().build();
assertEquals(a, b);
\end{minted}
\vfill
simplified example:
\vfill
\begin{minted}{java}
assertEquals("foo", "bar");
\end{minted}
\end{frame}

\section{Pop Quiz}

\begin{frame}[fragile]{What's the assert message?}
\begin{minted}{java}
assertEquals("foo", "bar");
\end{minted}
\vfill
\begin{itemize}[<+(1)->]
\item[a)] \texttt{expected: <foo> but was: <bar>}
\item[b)] \texttt{expected: <bar> but was: <foo>}
\item[c)] all of the above?
\item[d)] none of the above
\end{itemize}
\end{frame}

\begin{frame}[fragile]
\vfill
\begin{beamercolorbox}[sep=8pt,center,shadow=false,rounded=false]{title}
\usebeamerfont{title}The fact you had to think about it means \texttt{assertEquals()} is bad API design\par%
\end{beamercolorbox}
\vfill
\pause
\begin{minted}{java}
public static void assertEquals(Object expected, Object actual)
\end{minted}
({\footnotesize\url{https://junit.org/junit4/javadoc/4.12/org/junit/Assert.html}})
\end{frame}

\section{What does a good assert look like?}

\begin{frame}[fragile]{AssertJ > JUnit}
JUnit:
\begin{minted}{java}
assertEquals(expected, actual);
\end{minted}
\vfill
AssertJ:
\begin{minted}{java}
assertThat(actual).isEqualTo(expected);
\end{minted}
\vfill
(okay, the order still doesn't seem super obvious, but will be as soon as we start using other assertions)
\end{frame}

\begin{frame}{AssertJ > JUnit}
\begin{itemize}
\item obvious argument order
\item strongly typed; can't accidentally compare incompatible types
\item AssertJ assertions are chainable, very powerful, and extremely informative
\item you can write your own assertion helpers
\item better NPE information!
\item AssertJ seems more to type, but we'll get to that...
\end{itemize}
\end{frame}

\section{Still not convinced?}

\begin{frame}[fragile]{isEqualToIgnoringCase}
\begin{minted}{java}
assertTrue("foo".equalsIgnoreCase("bar"));
\end{minted}
\begin{verbatim}
...
\end{verbatim}
\vfill
\begin{minted}{java}
assertThat("foo").isEqualToIgnoringCase("bar");
\end{minted}
\begin{verbatim}
Expecting:
 <"foo">
to be equal to:
 <"bar">
ignoring case considerations
\end{verbatim}
\end{frame}

\begin{frame}[fragile]{hasSize}
\begin{minted}{java}
final List<String> list = ImmutableList.of("foo");
assertThat(list).hasSize(2);
\end{minted}
\begin{verbatim}
java.lang.AssertionError:
Expected size:<2> but was:<1> in:
<["foo"]>
\end{verbatim}
\vfill
you can actually see what's in the list!
\end{frame}

\begin{frame}[fragile]{NullPointerException}
\begin{minted}{java}
final List<String> list = null;
assertThat(list).hasSize(2);
\end{minted}
\begin{verbatim}
java.lang.AssertionError:
Expecting actual not to be null
\end{verbatim}
\vfill
it's a small win this isn't an NPE; but this does make triage easier.
\end{frame}

\begin{frame}[fragile]{More complex asserts!}
\begin{minted}{java}
assertThat(string)
    .startsWith("foo")
    .endsWith("bar");

assertThat(list)
    .hasSize(7)
    .contains(item);

assertThatExceptionOfType(ArithmeticException.class)
    .isThrownBy(() -> { ... })
    .hasMessageContaining("foo")
\end{minted}
\end{frame}


\section{Other learnings}


\begin{frame}[fragile]{JSONAssert}
\begin{itemize}
\item what about \texttt{org.skyscreamer.jsonassert.JSONAssert}?
\item can be good, can print absolutely useless errors in rare cases
\item hand-rolled JSON strings in Java \textit{suck} because of escaping; brittle
\item old version were worse printing no useful error
\end{itemize}
\pause
\vfill
\begin{minted}{java}
JSONAssert.assertEquals("{\"EXPECTED\": 1}", "\"ACTUAL\"", true);
\end{minted}
\begin{verbatim}
Expected: a JSON object
     got: org.skyscreamer.jsonassert.JSONParser$1@5d16055
\end{verbatim}
(okay, contrived, but still interesting)
\end{frame}


\section{TL;DR}


\begin{frame}[fragile]{TL;DR}
\begin{itemize}
\item AssertJ is really nice
\item obvious argument order
\item strongly typed
\item better NPE handling
\item easier to write complex asserts, more specific tests, less brittle
\item richer assertion messages means less time debugging, grok the problem quicker
\item prefer AssertJ over Hamcrest, because auto-complete makes discovering assertions easy, less nesting due to fluent style
\end{itemize}
\end{frame}


\section{Fin}


\end{document}
